\documentclass[12pt, a4paper]{article}

% =================================================================
% CORE SETUP (Encoding, Fonts, Layout)
% =================================================================
\usepackage{fontspec}
\setmainfont{Palatino Linotype} % Palatino font
%\setmainfont{TeX Gyre Pagella}
\linespread{1.3}                % Increased line spacing
\usepackage[margin=2.2cm]{geometry} % Page margins
\usepackage{ragged2e, setspace}  % Text alignment and spacing

% =================================================================
% GRAPHICS, TABLES, AND FIGURES
% =================================================================
\usepackage{graphicx}
% Essential for graphics and colors
\usepackage{soul}               % For highlighting text (can be slow, see notes)
\usepackage{adjustbox}          % Adjusting figure sizes
\usepackage{subcaption}         % For subfigures and subtables
\usepackage[justification=raggedright,singlelinecheck=false]{caption} % Caption customization
\usepackage{tabularx, multirow, booktabs, makecell} % Advanced tables
\usepackage{float, placeins}    % Better float positioning control

% =================================================================
% MATH, SYMBOLS, AND UTILITIES
% =================================================================
\usepackage{amsmath, amsfonts, amssymb, latexsym} % Standard math packages
\usepackage[inline]{enumitem}   % Better list customization
\usepackage[hang,flushmargin]{footmisc} % Footnote formatting
\usepackage{verbatim, multicol} % Multi-column support and verbatim text
\usepackage{dirtytalk}          % Proper quotation symbols

% =================================================================
% CITATIONS AND BIBLIOGRAPHY (CRITICAL FOR SPEED AND COMPILATION)
% =================================================================
\usepackage[style=apa, backend=biber]{biblatex}
\addbibresource{shocks_ineq.bib}

% =================================================================
% HYPERLINKS AND CUSTOM COMMANDS
% =================================================================
% Custom symbols
\def\sym#1{\ifmmode^{#1}\else\(^{#1}\)\fi}
\makeatletter

\def\@fnsymbol#1{\ensuremath{\ifcase#1\or \dagger\or \ddagger\or
   \mathsection\or \mathparagraph\or \|\or \P\or \dagger\dagger
   \or \ddagger\ddagger \else\@ctrerr\fi}}
\makeatother

\interfootnotelinepenalty=10000 % keeps large fotnotes on the same page

% HYPERREF
\usepackage[colorlinks=true, citecolor=blue, linkcolor=black, urlcolor=blue]{hyperref}


\title{Shocks and income inequality\thanks{Corresponding author: Oleg Gurshev. Email: o.gurshev@grape.org.pl.\\Financial support from the Polish National Science Centre (grant no. 2021/43/B/HS4/03241) is gratefully acknowledged.}}% grant statement
\author{
    Oleg Gurshev  \\ 
    \small{FAME$\mid$GRAPE} 
    %\footnotemark 
    \and 
    Lucas van der Velde \\ 
    \small{FAME$\mid$GRAPE} \\[-0.5em] 
    \small{Warsaw School of Economics}
}
\date{}

\begin{document}
%------------------------------------------------------------------------
\maketitle
\thispagestyle{empty}
% \renewcommand{\thefootnote}{}  % Suppress footnote numbering
% \footnotetext{$\dagger$ Corresponding author. Email: o.gurshev@grape.org.pl.\\Financial support from the Polish National Science Center (grant no. 2021/43/B/HS4/03241) is gratefully acknowledged.}
% \renewcommand{\thefootnote}{\arabic{footnote}} 
% Abstract
\begin{abstract}
\noindent 
% Oleg: changed confidence -> expectations
% Oleg: changed wording in the second sentence and added some additional description to reflect the newly put results
We examine the contribution of supply and demand shocks to income inequality in a panel setting. Leveraging the newly created Global Repository of Income Dynamics, we study the relationship between unanticipated supply and demand shocks and income inequality, distinguishing between international (U.S.) and domestic shocks. Our results show that shocks originating in the U.S., on average, increase income dispersion in other developed countries in a procyclical manner: a positive demand shock tends to produce stronger reactions than supply shock. Decomposing these effects reveals that shocks primarily affect the asymmetry of income changes rather than the overall level of income volatility. We explore different transmission channels—trade, financial and expectations. The trade channel appears particularly relevant for U.S. supply shock. Comparing these external shocks with domestic counterparts, we find that domestic demand shock exhibits similar dynamics to its U.S. counterpart, whereas a domestic supply shock is largely irrelevant.
\end{abstract}

\bigskip
\hspace*{15pt}\textbf{Keywords:} Inequality; Macroeconomic shocks; Administrative data
 
\hspace*{15pt}\textbf{JEL Classification:} J30, J31, E24, E32
\clearpage


% Done
% Intro - results + contribution (WIP done)
% Abstract - needs 1 sentence (WIP done)

% Todo list before Oct 2025:
% Results - result description (need description of 3 figures + revision of the older text) + possible framing related to the other studies
% Conclusion - need an update summary of the results

% if possible - add BE shocks somewhere (if appropriate)


%--------------------------------------------------------------------------------------
% List of changes - September 2025:
% - Added more recent studies that focus on US shock spillovers to the 2nd paragraph
% - Added more precise framing of the problem in paragraph 2
% - Revised some wording in paragraphs 1 and 3
% - Added additional justification to the use of BQ shocks in paragraph 4
% - Reworked contribution to be in two parts: inequality and shock transmission
% - Extra referencing done in the contribution
% - Revised the description of the results (summary)
% - Overall editing polish on articles and commas

% WIP revision - done
\section{Introduction}
% Introductory paragraph
It is now well recognized that the rise in economic inequality across advanced economies over past decades has multiple drivers\footnote{Including, \emph{inter alia}, technological progress \parencite{Bound1995, Acemoglu2002}, demographics \parencite{Karahan2013}, globalization \parencite{feenstra2003global}, labor market structure \parencite{jaumotte2015inequality}, and monetary policy \parencite{coibion2017innocent, furceri2018effects, Amberg2022, andersen2023monetary}.}. However, despite growing attention to the determinants of inequality, there is little systematic empirical evidence on how global shocks to supply and demand shape the entire income distribution, affecting not only overall inequality but also its underlying income dynamics.

% A short intro on demand and supply shocks and why domestic and US shocks can have different effects (with citations) 
At the same time, understanding the origins of international fluctuations continues to be a key area of research. Given the sheer scale and global influence of the United States (U.S.), its domestic macroeconomic changes are likely to have substantial implications for the global economy and its close economic partners \parencite{Kose2003, canova2005transmission, Kose2012, kose2017global, kalemli2013global, Fink2015, rey2016, ramey2016macroeconomic, miranda2022global, carrillo2020inquiry, levchenko2020tfp, lakdawala2021international, dees2021global, di2022international, lastauskas2023global}. The impact of these changes is often found to be heterogeneous, with the magnitude of the spillovers on other economies depending critically on their degree of economic integration with the U.S. This heterogeneity at the country level strongly suggests that the impact will also be uneven within countries. Yet, surprisingly little is known about the distributional consequences of these external shocks.

% Goal
This paper studies how income distributions react to supply and demand shocks originating in the U.S. and within national economies. To this end, we draw on a rich cross-country database: the Global Repository of Income Dynamics (GRID) by \textcite{guvenen2022global}. This database contains comparable moments of income distributions of unparalleled quality derived from administrative data. Our study analyzes data from countries that participated in the first phase of GRID and meet the minimum data requirements needed for the estimation of shocks: Canada, Denmark, France, Germany, Italy, Mexico, Norway, Spain, and Sweden.

% Methodology
Our analysis proceeds as follows. First, we estimate supply and demand shocks using long-run restrictions\footnote{See characterization of popular identification strategies in \textcite{ramey2016macroeconomic}.} as proposed by \textcite{blanchard1989dynamic}\footnote{This seminal paper has been revisited by \textcite{Binet2015, Herwartz2018, Keating2013}.}. We adopt this approach because its data requirements are minimal, making it particularly suitable for our broad international setting. Specifically, this method imposes restrictions based on economic theory, where supply shocks are assumed to have permanent effects on output, while demand shocks have only temporary effects. We later test the sensitivity of our findings to the type of U.S. demand shock being used by estimating an alternative demand shock using the seminal approach of \textcite{bayoumi1992shocking}. The second step involves estimating the response of income dispersion to U.S. and country-specific (domestic) shocks using impulse response functions (IRFs) estimated directly from local projections \parencite{jorda2005estimation, jorda2024local}. In this step, we also study the reaction of the income distribution, measured by the standard deviation and the Kelley skewness of the residual first-year log income changes, to these shocks. Finally, we study the three potential transmission channels that are frequently identified in the literature: trade linkages \parencite{corsetti2011multilateral}, financial market integration \parencite{faccini2016international}, and expectations \parencite{klein2021real} using state-dependent local projections in the style of \textcite{auerbach2013output}.

% Result description - WIP
Our findings indicate that supply and demand shocks originating in the U.S. tend to raise income dispersion abroad. We also confirm that these changes are largely procyclical. Decomposing these dynamics reveals that shocks primarily alter the asymmetry of income changes rather than the overall level of income volatility. While all shocks make the income distribution more positively skewed, the effect on volatility reveals a critical distinction: U.S. supply shock increases volatility, whereas domestic supply shock acts as a stabilizing force by significantly decreasing it. When considering transmission channels, the distinction between demand and supply shocks is relevant. Demand shock increases inequality regardless of the level of exposure. By contrast, domestic supply shock produces more heterogeneous response.


%The difference in responses is consistent with models of= international trade, where firms participating in international markets are more productive and offer goods of a higher quality. Demand shocks in the US are transmitted to higher wages among the most productive firm abroad, i.e. an export supply channel. We find evidence in favor of this by using responses to US shocks together with trade exposure interaction%.

% Contribution - WIP
This paper makes two main contributions. First, our findings complement the recent body of studies that investigate the dynamic causal link between macroeconomic shocks and the Gini such as \textcite{coibion2017innocent, Davtyan2017, furceri2018effects} by providing novel evidence using a set of well-established shocks. Specifically, our analysis goes beyond documenting the impact of shocks on the Gini coefficient, as we also examine their possible effects on the distributional measures using one-year residual log income changes: standard deviation and Kelley skewness, revealing new patterns that have so far received little attention in prior research. Second, we report new evidence related to the transmission of U.S. supply and demand shocks abroad via trade, financial, and expectations channels. This is done to identify how external shocks affect inequality through changes in export demand, credit supply, and firm's expectations about future economic activity. Here, our findings complements the growing literature that studies spillover effects and transmission of various shocks originating within the US: \textcite{canova2005transmission, mackowiak2007external, akinci2013global, bowman2015us, dedola2017if, carrillo2020inquiry, levchenko2020tfp, di2022international, azad2022spillovers, lastauskas2023global, lastaukas2024}. We document the critical importance of all three channels when it comes to U.S. supply shocks. Countries with strong export links to the U.S. tend to experience a significant and lasting rise in inequality. In contrast, countries with high financial exposure experience only a brief increase in inequality immediately following the shock, while those with weaker financial exposure see a gradual rise even after the three-year horizon. Finally, lower domestic business confidence corresponds to a stronger inequality response.

%The comparison between domestic and international shocks reveals fundamental differences. First, domestic shocks generate weaker, and often not statistically significant, responses. Second, domestic supply shocks are associated with a decline in inequality.

The paper is structured as follows. Section 2 describes data and empirical methodology. Section 3 reports the results. Section 4 concludes.

%-----------------------------------------------------------------------------------
% List of changes - September 2025:
% - Changed section and subsection names to better reflect content
% - Moved Tables from Appendix to main text
% - Added necessary text descriptions
% - Added VAR algebra
% - Revised Table 1 to include Kelley and Std

% WIP revision - done
\section{Methodology}

\subsection{Inequality measures and shocks} 
The Global Repository of Income Dynamics (GRID) provides measures of inequality from administrative records across several countries. This source has several advantages. First and foremost, income is less subject to reporting errors, and there is an adequate representation of earners at the top of the income distribution, neither of which is not guaranteed in other databases. Second, estimates are based on larger samples, quite often the entire working population. Finally, GRID also provides better coverage than similar open source databases (OECD, Luxembourg Income Study), as time series are uninterrupted. However, the database has some limitations, namely: i) income refers to labor income at the individual level, ii) since it is based on tax records, envelope payments are not included. As our sample contains mostly developed countries, the bias introduced might not be significant.

All income inequality measures are computed only among individuals between ages 25-55, who are expected to be active in the labor market. To ensure that individuals are attached to the labor markets, the sample used in GRID is further restricted to those perceiving yearly earnings above a minimum threshold (one fourth of the minimum wage). All measures are based on gross earnings\footnote{Each country has it's own specific approach to measuring gross earnings. However, the resulting measures are comparable as they include all forms of compensation subject to taxation and social security contributions (i.e., base salary, overtime compensation, performance and seasonal bonuses, paid vacations, paid sick leaves, and severance payments).} deflated to 2018 price levels. Table \ref{table:a5} presents descriptive statistics (means) for the Gini coefficient together with distributional measures of residual one-year log income changes (standard deviation and Kelley skewness) as collected from GRID.

We recover supply and demand shocks using the long-run restrictions approach pioneered by \textcite{blanchard1989dynamic}. The identification of shocks begins with a reduced-form VAR of order $p$:
\begin{equation} \label{eq:var_reduced}
X_t = \sum_{i=1}^{p} A_i X_{t-i} + e_t
\end{equation}
where $X_t = [\Delta y_t, u_t]'$ is the vector of endogenous variables (growth rate of real output and the unemployment rate), $A_i$ are coefficient matrices, and $e_t$ is a vector of serially uncorrelated reduced-form residuals with covariance matrix $\Omega$.

These reduced-form residuals are linear combinations of the underlying structural shocks, $\epsilon_t = [\epsilon^s_t, \epsilon^d_t]'$, which represent supply and demand shocks, respectively. The relationship is given by:
\begin{equation} \label{eq:shocks_relation}
e_t = S\epsilon_t
\end{equation}
where we assume the structural shocks are orthonormal, i.e., $E[\epsilon_t \epsilon_t'] = I$. To identify the matrix $S$, we consider the moving-average representation of the VAR, $X_t = C(L)e_t = C(L)S\epsilon_t$. The long-run impact of the structural shocks on the variables is given by the matrix $C(1)S$.

The key identifying assumption is that demand shocks ($\epsilon^d_t$) have no long-run effect on the level of output. This economic restriction implies that the cumulative effect of a demand shock on the output growth rate, $\Delta y_t$, must sum to zero. This forces the $(1,2)$ element of the long-run multiplier matrix to be zero, making the matrix lower triangular. This constraint, combined with the condition from the covariance matrix ($SS' = \Omega$), provides the necessary restrictions to uniquely identify the structural shocks.

% Table: panel data coverage (revised variant)
\begin{table}[H]
\captionsetup{justification=raggedright, singlelinecheck=false}
\centering
\caption{Availability of GRID data.}
\small 
\setlength{\tabcolsep}{5pt} 
\renewcommand{\arraystretch}{0.95}
\begin{tabular}{l c c c c}
\toprule
Country & Scope & Gini & Std & Kelley skewness \\
\midrule
Canada   & 1990--2019 & 0.41 (0.01) & 0.53 & 0.02\\
Denmark  & 1990--2016 & 0.28 (0.01) & 0.42 & 0.03\\
France   & 1991--2016 & 0.34 (0.00) & 0.47 & -0.03\\
Germany  & 2001--2016 & 0.40 (0.01) & 0.40 & 0.18\\
Italy    & 1990--2016 & 0.36 (0.02) & 0.48 & 0.03\\
Mexico   & 2005--2019 & 0.56 (0.00) & 0.65 & -0.01\\
Norway   & 1993--2017 & 0.33 (0.01) & 0.59 & -0.01\\
Spain    & 2005--2018 & 0.40 (0.01) & 0.50 & 0.02\\
Sweden   & 1990--2016 & 0.30 (0.01) & 0.49 & 0.04\\
\bottomrule
\end{tabular}

\vspace{0.1cm}
\parbox{0.9\linewidth}{\raggedright\footnotesize Note: own summary. Scope refers to the availability of Gini data. The panel is unbalanced, with a total of $N=217$ country-year observations for the Gini coefficient. Gini is reported as mean (standard deviation). Kelley skewness and standard deviation are from residual one-year log income changes. Their effective coverage is one year shorter at both the start and end of the sample compared to the Gini series. All data are annual.}
\label{table:a5}
\end{table}



Following  this framework, we estimate a bivariate VAR for each country using quarterly rates of unemployment and real output growth\footnote{Lag length is selected using SC separately for each country: one lag (Canada, Italy, Mexico, Norway), two lags (Denmark, France, Germany, Spain Sweden, USA). Impulse response functions for each country (demand and supply shocks) are available in Figures \ref{fig:var_impulses_demand} and \ref{fig:var_impulses_supply} (Appendix B). While demand shocks are temporary, they decay at a slow rate. In some countries, the responses are different from zero even 20 quarters after the initial shock (see Figure \ref{fig:var_impulses_demand} in Appendix B).}. We collect the necessary data from the Federal Bank of St. Louis (FRED) and the OECD databases.\footnote{Even if data requirements are minimal, they are not satisfied by every country. Argentina and Brazil lack data on unemployment rates for the early years of the sample. Therefore, we excluded these countries from further analysis.}. All series were de-meaned prior to VAR input. Detailed description of the data used for the estimation of the bivariate models is available in Table \ref{table:a1} (Appendix A). Tables \ref{table:a2} and \ref{table:a3} display the correlation of quarterly supply and demand shocks across countries. Shocks generally feature low degree of correlation across countries except the three pairs (DEU-ESP, DEU-FRA, DEU-MEX). Finally, given that GRID data are available at the yearly level, we annualize (average within each year) and standardize (mean-center and scale to unit variance) the obtained shocks before using them in panel estimation. This transformation ensures comparability across countries, prevents scale effects from biasing the estimates, and facilitates interpretation of the impulse responses in standard deviation units.

% Table A2 supply shock
\begin{table}[H]
\captionsetup{justification=raggedright,
singlelinecheck=false
}
    \centering
    \caption{Pairwise correlations: supply shock.}  
    \begin{tabular}{l ccc ccc ccc c}
    \toprule
 & CAN & DEN & DEU & ESP & FRA & ITA & MEX & NOR & SWE & USA \\ 
  \hline
  CAN & 1 &  &  &  &  &  &  &  &  &  \\ 
  DEN & -0.08 & 1 &  &  &  &  &  &  &  &  \\ 
  DEU & -0.24 & 0.14 & 1 &  &  &  &  &  &  &  \\ 
  ESP & -0.16 & 0.05 & 0.38 & 1 &  &  &  &  &  &  \\ 
  FRA & 0.07 & 0.08 & -0.25 & -0.22 & 1 &  &  &  &  &  \\ 
  ITA & -0.14 & 0.01 & 0.1 & -0.06 & 0.17 & 1 &  &  &  &  \\ 
  MEX & -0.19 & -0.13 & 0.4 & 0.2 & -0.08 & 0.15 & 1 &  &  &  \\ 
  NOR & 0.14 & 0.12 & -0.04 & 0 & -0.02 & 0.05 & 0.02 & 1 &  &  \\ 
  SWE & -0.04 & 0.14 & 0.31 & 0.25 & -0.01 & 0.1 & 0.02 & 0.13 & 1 &  \\ 
  USA & 0.02 & 0.17 & 0.09 & 0.22 & -0.08 & 0.07 & 0.18 & 0.03 & 0.12 & 1 \\ 
  \bottomrule
    \end{tabular}
    \begin{minipage}{\textwidth}
    \vspace{0.1cm} 
    \footnotesize Note: own summary, shocks are obtained using long-run restrictions. The period under analysis is 1990:Q2-2019:Q3 for all countries except Germany (1991:Q2-2019:Q3).
    \end{minipage}
    \label{table:a2}
\end{table}

%-----------------------------------------------------------------------------

% Table A3 demand shock
\begin{table}[H]
\captionsetup{justification=raggedright,
singlelinecheck=false
}
    \centering
    \caption{Pairwise correlations: demand shock.}
    \begin{tabular}{l ccc ccc ccc c}
    \toprule
& CAN & DEN & DEU & ESP & FRA & ITA & MEX & NOR & SWE & USA \\ 
  \hline
  CAN & 1 &  &  &  &  &  &  &  &  &  \\ 
  DEN & 0.2 & 1 &  &  &  &  &  &  &  &  \\ 
  DEU & 0.19 & 0.05 & 1 &  &  &  &  &  &  &  \\ 
  ESP & 0.12 & -0.03 & 0.1 & 1 &  &  &  &  &  &  \\ 
  FRA & 0.24 & 0.17 & 0.36 & -0.01 & 1 &  &  &  &  &  \\ 
  ITA & 0.14 & 0.2 & 0.14 & -0.07 & 0.28 & 1 &  &  &  &  \\ 
  MEX & 0.22 & 0.17 & 0.12 & 0.15 & 0.12 & 0.17 & 1 &  &  &  \\ 
  NOR & 0.13 & 0.23 & -0.11 & -0.02 & 0.13 & 0.14 & 0.1 & 1 &  &  \\ 
  SWE & 0.23 & 0.15 & 0.15 & -0.03 & 0.2 & 0.17 & 0.07 & 0.11 & 1 &  \\ 
  USA & 0.16 & 0.23 & 0.16 & -0.07 & 0.12 & 0.25 & 0.16 & 0.16 & 0.05 & 1 \\ 
    \bottomrule
    \end{tabular}
    \begin{minipage}{\textwidth}
    \vspace{0.1cm}
    \footnotesize Note: own summary, shocks are obtained using long-run restrictions. The period under analysis is 1990:Q2-2019:Q3 for all countries except Germany (1991:Q2-2019:Q3).
    \end{minipage}
    \label{table:a3}
\end{table}


\subsection{Local projections} 
To study the impact of supply and demand shocks on (the level of) inequality, we compute cumulative IRFs directly from local projections. Specifically, we estimate the following regression at the country level:
\begin{equation}
    y_{c,t+h}-y_{c,t-1} =  \beta^h z_{c,t} + \gamma^h_c + \gamma^h_t + \pi^h X_{c,t} + e^h_{c, t+h}
\label{eq:1}
\end{equation}
\noindent where $y_{c,t+h}$ is the log of Gini for country $c$ measured at time $t+h$, $z_{c,t}$ is the exogenous shock, and $\beta^h$ are the estimated responses for $h = 0,..,3$ periods after the shock. The term $\gamma^h_c$ accounts for country-specific fixed effects, while $\gamma^h_t$ control for period-specific differences. For domestic shocks, $\gamma^h_t$ corresponds to time fixed effects. For shocks originating in the U.S. (which affect all countries), $\gamma^h_t$ captures NBER-identified recessions (including the level and two lags).

Our baseline set of controls ($X_{c,t}$) includes two lags of: changes in inequality ($\Delta y_{c,t-i}$, for $i =1,2$) and exogenous shock used ($z_{c,t-i}$, for $i =1,2$), i.e. supply or demand. As a robustness check, we expand the set of control variables to include two lags of: i) share of exports to the U.S. to total exports (trade exposure), ii) share of U.S. bank claims to GDP (financial exposure), iii) changes in \textit{de facto} economic openness (proxied by the \textit{de facto} component of the KOF index), iv) expectations (proxied by the OECD's business confidence index), and v) changes in domestic labor market policies (proxied by the Economic Freedom of the World's indicator of labor market regulation), see Table \ref{table:a4} for details (Appendix A).

To examine the three potential transmission channels of supply and demand shocks originating in the U.S. we apply the state-dependent local projection in the style of \textcite{auerbach2013output}. Namely, we estimate the following regression:
\begin{equation}
    y_{c,t+h}-y_{c,t-1} = \beta^h z^{US}_{t} + \delta^h (z^{US}_{t} \times s_{c,t-1}) + \pi^h X_{c,t} + \gamma^h_c + e^h_{c, t+h}
\label{eq:2}
\end{equation}
where $s_{c,t-1}$ represents the state variable: i) percentage of exports to U.S. in all exports of country $c$ (trade channel), ii) bilateral U.S. bank claims as a proportion of GDP in country $c$ (financial channel), or iii) business confidence in country $c$ (expectations channel). $X_{c,t}$ includes two lags of: changes in inequality, exogenous shock being used, interaction term, state variable, and NBER-identified recessions. The state-dependent cumulative impulse response is the linear combination $\beta^h + \delta^h \times s_{c,t-1}$.

Finally, all estimations use Driscoll-Kraay standard errors in reporting confidence bands. These standard errors accommodate different forms of autocorrelation and heteroskedasticity.

%-------------------------------------------------------------------------------------
% List of changes - September 2025:
% - Split into two subsections
% - Added results for asymetric impact for US shocks
% - Moved subsample figure from Appendix to main text
% - Added additional results for Kelly and Std
% - Added economic narrative
% - Revised text to be more coherent

\section{Results}
We report our results as follows. First, we estimate the baseline responses of the log of Gini to unanticipated, one-standard-deviation change in a U.S. or domestic shock. Next, we extend our baseline analysis to two measures of the income distribution. Namely, we replace the Gini with: i) the standard deviation of one-year log income changes, which captures the overall income volatility, and ii) the Kelley skewness of the same log income changes, which measures the asymmetry of income risk. Second, we present the results for the transmission channels of U.S. shocks. We motivate our investigation into channels by first examining the asymmetric impact of U.S. shocks, and then present the results from the state-dependent local projections. Finally, we assess the sensitivity of our findings to alternative demand shocks. Namely, we compare our original responses with added controls to the responses obtained using an alternative approach of \textcite{bayoumi1992shocking} that uses inflation as a proxy for a U.S. demand shock.

%Finally, we check robustness of our baseline estimates by including additional controls related to alternative drivers of inequality.

%Finally, we report results using two subpopulations: men and women. 
\subsection{Inequality}
% US supply and demand shocks
The upper row of Figure \ref{fig:demand_supply_base} displays the responses of the log of Gini to demand and supply shocks originating in the U.S. A demand shock leads to a significant and long-lasting increase (up to 60 basis points at the 1\% level) in income inequality. Supply shock produces much smaller increases (up to 40 basis points at the 1\% level). Domestic demand shock generates IRF of around 30 basis points, as shown in the bottom row of Figure \ref{fig:demand_supply_base}. This shock produces an initial hike that quickly vanishes, whereas domestic supply shock is insignificant at all horizons. Overall, domestic demand shock increases inequality, though responses remain lower than those to U.S. shocks. 

%Moreover, estimates from the intermediate specification suggest that responses are driven (partly) by sample composition. 

% Standard deviation
Figure \ref{fig:std_base} illustrates the impact of four shocks on the standard deviation of residual one-year log income changes. Both U.S. shocks eventually increase income dispersion, but they propagate differently. The response to the U.S. demand shock is statistically insignificant for the first two periods but becomes positive and significant by period $t+3$, suggesting that the inequality effects of U.S. demand take time to materialize. The U.S. supply shock causes a statistically insignificant compression but leads to a significant and persistent increase in the standard deviation from period $t+2$ onward. In contrast, domestic shocks generate substantially different responses. Specifically, we observe an insignificant reaction to a domestic demand shock, whereas a domestic supply shock tends to narrow the distribution, reducing income dispersion.

% Kelley
Next, Figure \ref{fig:kelley_base} presents the impact of shocks on the Kelley skewness of residual one-year log income changes. A U.S. demand shock leads to a significant increase in right-skewness, suggesting it disproportionately stretch the upper tail of the income distribution. The U.S. supply shock also generates an increase in skewness, but has a short-lived effect that dissipates by period $t+2$. Responses to domestic shocks are more pronounced and follow an opposing pattern. A domestic demand shock causes a persistent and significant decrease in skewness, making the distribution more left-skewed, whereas a domestic supply shock generates a steady and significant increase in right-skewness over the entire horizon.

% Economic narrative
Taken together, these results suggest that U.S. shocks play a key role in shaping domestic income inequality relative to domestic disturbances observed in our sample. This finding is in line with the existing evidence on economic co-movement between booms and busts occurring in the U.S. and the rest of the world \parencite{Kose2003, Kose2012, Fink2015}. Further, the effect of a U.S. demand shocks is both large and persistent, consistent with the idea that stronger U.S. demand raises export opportunities and capital returns abroad, thereby widening inequality. In contrast, U.S. supply shocks generate smaller but more broadly distributed effects, while domestic shocks remain modest and often work in the opposite direction. The evidence on skewness further indicates that foreign shocks primarily operate by stretching the upper tail of the income distribution.

We conduct a number of robustness exercises. First, we evaluate the evolution of responses beyond the initial estimation horizon of three years, see Figure \ref{fig:irf_longer} (Appendix B). Given that the panels are short, the obtained estimates are less reliable, which is reflected in the broader confidence bands. To the extent that conclusions are possible, the response to a U.S. demand shock decreases over time, whereas a U.S. supply shock produces more persistent responses. Further, we check whether the inclusion of additional drivers of the Gini coefficient or specific channels affects the estimated responses as mentioned in the equation (\ref{eq:1}). The resulting IRFs are portrayed in Figure \ref{fig:demand_supply_robust} (Appendix B). The patterns described for U.S. demand shock are robust to the inclusion of these controls. The trajectory of responses to U.S. supply shock is also identical, but shifted downwards. Since the additional controls are not available each year we also include an intermediate specification, which restricts the sample, but does not include any of the additional control variables (see Table \ref{table:a8} in Appendix B for the estimates we obtain using controls and additional restrictions). Finally, we include the same set of controls and estimate responses for the standard deviation and Kelley skewness (see Figures \ref{fig:std_robust} and \ref{fig:kelley_robust} in Appendix B). When we swap out the standard deviation for the 90-10 percentile difference, the overall shape of the response obtained from standard deviation is preserved (see Figure \ref{fig:9010_demand_supply} in Appendix B).

% Baseline result
\begin{figure}[H]
    \centering    
    \caption{Cumulative impulse responses to demand and supply shocks: Gini, baseline.}  
    \label{fig:demand_supply_base}
    \includegraphics[width=0.75\textwidth]{Figures/baseline_demand_supply_LP_extended.pdf}
    \centering \caption*{Note: shaded areas represent 68\% Driscoll-Kraay confidence bands. Detailed output of our baseline result is available in Table \ref{tab:a7_baseline_reg_output} (Appendix B).}
\end{figure}

% IRFs for Std
\begin{figure}[H]
    \centering    
    \caption{Cumulative impulse responses to demand and supply shocks: standard deviation, baseline.}  
    \label{fig:std_base}
    \includegraphics[width=0.75\textwidth]{Figures/std_demand_supply_LP.pdf}
    \centering \caption*{Note: shaded areas represent 68\% Driscoll-Kraay confidence bands.}
\end{figure}

% IRFs for Kelley
\begin{figure}[H]
    \centering    
    \caption{Cumulative impulse responses to demand and supply shocks: Kelley skewness, baseline.}  
    \label{fig:kelley_base}
    \includegraphics[width=0.75\textwidth]{Figures/kelley_demand_supply_LP.pdf}
    \centering \caption*{Note: shaded areas represent 68\% Driscoll-Kraay confidence bands.}
\end{figure}


%Estimations are noisier, though, and any conclusion should be taken with some salt.      %We interpret the behavior of the demand shock by the existing inter-market dependencies across our sample, where changes in US demand have a strong impact abroad. As employment is linked with income and current consumption, we think US consumption of foreign goods and services is the main driver behind these responses. This finding is also in line with the existing evidence on economic co-movement between booms and busts occurring in the US vis-a-vis the world \parencite{Kose2003, Kose2012, Fink2015}. 

\subsection{Transmission channels of U.S. shocks}
% Asymmetric shocks
We begin by investigating whether positive and negative U.S. shocks generate asymmetric responses. As Figure \ref{fig:demand_supply_asym} shows, the Gini responds in a procyclical manner, particularly to supply shocks, while negative demand shocks have negligible effects. We treat this asymmetry as indicative of the fact that distributional outcomes can potentially be explained by the specific transmission channels we study below.

% Asymetric shocks
\begin{figure}[H]
    \centering
    \caption{Cumulative impulse responses to positive and negative U.S. demand and supply shocks: Gini.}
    \label{fig:demand_supply_asym}
    \includegraphics[width=0.65\textwidth]{Figures/asymmetric_IRFs.pdf}
    \centering \caption*{Note: shaded areas represent 68\% Driscoll-Kraay confidence bands.}
\end{figure}

Our analysis of the three transmission channels: trade linkages \parencite{corsetti2011multilateral}, financial market integration \parencite{faccini2016international}, and expectations \parencite{klein2021real} supports this view, especially for supply shocks (see responses in Figure \ref{fig:demand_supply_channels_ag}). Countries with strong export links with the U.S. experience a large and persistent increase in inequality. Financial integration also plays a complex role: tightly integrated countries see a sharp but short-lived rise in inequality, whereas those with weaker links experience a more gradual but sustained increase over the estimation horizon. Finally, lower domestic business confidence is associated with a more pronounced inequality response.

In contrast, the transmission of demand shocks appears more nuanced. In our initial state-dependent specification, none of these channels explain the variation in responses. However, a data-driven subsample analysis reveals a distinction (Figure \ref{fig:demand_supply_channels}). When we split countries by their median level of exposure, the role of this channel becomes evident. Countries with high trade integration: Canada, Germany, Italy, Mexico, and Sweden exhibit a significantly stronger inequality response to U.S. demand shocks relative to their less-integrated peers in our sample. This suggests the initial asymmetry is partly explained by the fact that supply shocks activate multiple channels broadly, while the transmission of demand shocks is more narrowly concentrated in highly trade-exposed economies.


% Can the result be framed in relation to the other studies?

%splitting the panel between countries that are more (less) exposed to US shocks. In the case of trade, we split the samples based on the percentage of exports to the US as a percentage of all exports\footnote{Canada, Germany, Italy, Mexico,
%Sweden belong to the group of more exposed countries; Denmark, France, Norway and Spain are the least exposed.}. In the case of financial linkages, we split the sample based on  US bank claims in a given country as a percentage of its GDP \footnote{Canada, Denmark, France, Germany, and Mexico belong to the group of more exposed countries;  Italy, Norway,
%Spain, and Sweden are the least exposed.}. 

%The results, plotted in Figure     \ref{fig:demand_supply_channels}, suggest that the trade channel is particularly relevant. More exposed countries react more strongly to both demand and supply shocks in the US. The reaction to demand shocks follows expectations from recent trade literature \parencite{furusawa2019international,adao2022imports}. An increase US demand for foreign goods leads to a reallocation of production towards exporting firms in foreign economies. As these firms tend to pay better, and have a more productive workforce, these changes favor higher earners. The scale of US demand shocks is large due to its sheer economic size \parencite{kose2017global}. Notice that among countries with lower exposure the IRF to US demand shocks are weaker and of an opposite signs. Only as of the third period we observe a reversal of signs, which suggest that the export-led channel might be at work with some lags. In the case of supply shocks, the response is also stronger among countries that have closer ties to the US. As in the previous case, inequality increases. This can reflect an import channel, where imports from the US displace domestic workers in less competitive firms. 

%The second split distinguishes between countries based on their exposure to US claims. Our initial expectation is that more exposed countries would present a stronger reaction, as a negative shocks could pull investments away from foreign countries. Yet, estimated IRF's present similar trajectories, and their confidence intervals present substantial overlap. 

% This part is not okay
%On the other hand, supply shocks would be expected to increase income inequality, as import exposure has been found to be pro-rich \parencite{adao2022imports}. And we do have some evidence in that direction. However, the effects tend to be small, as one would expect if a part of the surplus shock is absorbed by the domestic market (in this case, the US market). This is not the only relevant margin. Our specifications capture variation (even if imperfectly) on trade intensity, and trade exposure to the US, and the estimated impulse responses remain in the same ballpark. This points to other channels, such as co-movement between booms and busts between US and the rest of the world \parencite[see the discussions in][]{Kose2003, Kose2012, Fink2015}.



% Domestic shocks
% Next, we turn to the estimation of responses of domestic inequality to domestic shocks. %As was explained above, the extent to which demand shocks originating in the US can affect inequality abroad is linked with import demand (purchasing power and market size). As a result, US demand shocks can have a greater impact than domestic (local) shocks on foreign workers because the purchasing power and scale of US consumer market exceeds that of each country \parencite{kose2017global}. On the other hand, US supply shocks can represent a major (and diverse) supply chain disruptions as US multinational firms rely heavily on foreign inputs \parencite{moran2016offshoring}, whereas domestic supply shocks are usually more regional \parencite{Kose2012}. 
% Bottom row of Figure \ref{fig:demand_supply_base} reports our baseline results. The estimated effect of domestic demand shock leads to a large and long-lasting increase in income inequality (up to 15 basis points). The effect of the shocks is in the same direction as in the case of a US demand shock, but the effect is more muted. Within a trade framework, this arises because the local shocks increase demand for all products (though to a different extent), whereas foreign shocks bolster the demand for higher quality products. Domestic supply shocks tend to decrease income inequality. However, the effects are not statistically significant and of a much smaller magnitude.
 


% Gender
%As another robustness, we study whether Gini coefficients obtained from subpopulations exhibit similar responses. Figures \ref{fig:demand_supply_gender_base} and  \ref{fig:demand_supply_gender_robust} (Appendix) mirror the results obtained for the entire population with a noticeable gap between two subpopulations for the US demand shock, income inequality grows faster for men, than for women, which can be attributed to the variability in earnings among men in our sample (see Table \ref{table:a5} for reference). 



% Auerbach - Gorodnichenko state-dependency
\begin{figure}[H]
    \centering
    \caption{Cumulative state-dependent impulse responses to U.S. demand and supply shocks: Gini.}
    \label{fig:demand_supply_channels_ag}
    \includegraphics[width=0.80\textwidth]{Figures/state_dependent_IRFs}
    \centering \caption*{Note: levels are data-driven, i) exports (weak: up to 50th percentile; strong: 90th percentile), ii) bank claims (weak: 25th percentile, strong: 75th percentile), iii) business confidence (low: 25th percentile, high: 75th percentile). Shaded areas represent 68\% Driscoll-Kraay confidence bands.}
\end{figure}

% Channels - Oleg: this is generated using subsample_channels.r script
\begin{figure}[H]
    \centering    
    \caption{Cumulative impulse responses to demand and supply shocks: transmission channels of U.S. shocks across subsamples.}    
    \label{fig:demand_supply_channels}
    \includegraphics[width=0.75\textwidth]{Figures/high_low_LP_channels_US.pdf}
   \centering \caption*{Note: red response represents \say{high} subsample, blue response represents \say{low} subsample. Sample splitting is done using pooled country-level medians of each measure. Sample composition: i) exports (high exposure: Canada, Germany, Italy, Mexico, Sweden; low exposure: Denmark, France, Norway, Spain), ii) bank claims (high exposure: Canada, Denmark, France, Germany, Mexico; low exposure: Italy, Norway, Spain, Sweden), iii) business confidence (high confidence: France, Italy, Mexico, Norway, Spain; low confidence: Canada, Denmark, Germany, Sweden). Shaded areas represent 68\% Driscoll-Kraay confidence bands.}
\end{figure}

% Idea for subsection 3.3
\subsection{Sensitivity to an alternative U.S. demand shock}
\label{section:sensitivity}
In this subsection, we test the sensitivity of our original results related to U.S. shocks by estimating a bivariate VAR for the U.S. economy using de-meaned quarterly growth rates of real GDP and the seasonally adjusted CPI following \textcite{bayoumi1992shocking} (BE). This provides us with a systematic way to assess the extent to which our original results depend on the type of demand shock being used. Notably, the two U.S. demand shocks are quite different from each other, sharing only a 0.10 correlation. We input the estimated shocks into equation (\ref{eq:1}) and compare the resulting cumulative IRFs of Gini, standard deviation, and Kelley skewness (with all robustness controls for channels, changes in economic openness, and changes in domestic labor market policies) across the two sets of U.S. shocks (BQ and BE).

Figure \ref{fig:bqbe_gini} plots the responses of log of Gini to BQ shocks (the upper row) and BE shocks (the bottom row). Overall, we find the results highly symmetric (both in shape and magnitude). We confirm that a U.S. demand shock, on average, leads to a statistically significant increase in the Gini, whereas a U.S. supply shock has an insignificant impact once all controls are included relative to the baseline estimation results.

Next, we report the results for standard deviation and Kelley skewness. Figure \ref{fig:bqbe_std} presents the responses for income dispersion. Here, the impact of a U.S. demand shock carries notable differences: the BQ demand shock suggests a significant short-run decrease in dispersion, while the BE demand shock finds a delayed but large increase. The impact of both U.S. supply shocks remains relatively symmetric. The results for Kelley skewness, shown in Figure \ref{fig:bqbe_kelley}, are also nuanced. The finding that U.S. supply shocks robustly cause initial right-sided skewness is confirmed across both sets of shocks. In stark contrast, the effect of a U.S. demand shock is not symmetric. Specifically, the BE demand shock closely mirrors its supply counterpart, whereas the BQ demand shock resembles our baseline result, producing more asymmetry with a notable delay rather than initially. 

% Side-by-side comparison between BE and BQ shocks with controls
% Figure - US only -> Gini
\begin{figure}[H]
    \centering    
    \caption{Cumulative impulse responses to BE and BQ shocks: Gini, all controls.}  
    \label{fig:bqbe_gini}
    \includegraphics[width=0.75\textwidth]{Figures/robust_BQBE_demand_supply_LP.pdf}
    \centering \caption*{Note: shaded areas represent 68\% Driscoll-Kraay confidence bands.}
\end{figure}

% Figure - US only -> Std
\begin{figure}[H]
    \centering    
    \caption{Cumulative impulse responses to BE and BQ shocks: standard deviation, all controls.}  
    \label{fig:bqbe_std}
    \includegraphics[width=0.75\textwidth]{Figures/robust_std_BQBE_demand_supply_LP.pdf}
    \centering \caption*{Note: shaded areas represent 68\% Driscoll-Kraay confidence bands.}
\end{figure}

% Figure - US only -> Kelley
\begin{figure}[H]
    \centering    
    \caption{Cumulative impulse responses to BE and BQ shocks: Kelley skewness, all controls.}  
    \label{fig:bqbe_kelley}
    \includegraphics[width=0.75\textwidth]{Figures/robust_kelley_BQBE_demand_supply_LP.pdf}
    \centering \caption*{Note: shaded areas represent 68\% Driscoll-Kraay confidence bands.}
\end{figure}
% Gender 
%\begin{figure}[H]
%    \centering    
%    \caption{Cumulative impulse responses to demand and supply shocks: Gini %(by gender), baseline.}    
%    \label{fig:demand_supply_gender_base}
%    \includegraphics[width=0.85\textwidth]{Figures/baseline_gender_LP_extended.pdf}
%    \centering \caption*{Note: grey response represents men, pink response represents women, shaded areas are 68\% Driscoll-Kraay confidence bands. }
%\end{figure}

%-----------------------------------------------------------------------------
% List of changes
% - New conclusion
% - New findings
% - Retained old paragraph about GRID
\newpage
\section{Concluding remarks}
This paper investigated the relationship between a broad set of macroeconomic shocks and income inequality using local projections. In line with the existing empirical studies on international shock spillovers, our results also establish a clear hierarchy - shocks originating in the U.S. are the most potent drivers of aggregate inequality abroad, with effects that are larger and more persistent than those of domestic shocks. We find that positive U.S. demand shocks, in particular, robustly increase the Gini coefficient.

The core contribution of our analysis, however, is to move beyond this aggregate view. We demonstrate that a single inequality metric is often insufficient, as different shocks reshape the income distribution in fundamentally different ways. By analyzing higher-order moments, we uncover the more nuanced characteristics of each shock. A U.S. demand shock conforms to a classic narrative of rising inequality, increasing overall dispersion by stretching the upper tail of the income distribution. Domestic shocks, while smaller in aggregate, induce a profound and complex internal reshuffling. A domestic supply shock, for example, narrows the overall distribution but simultaneously increases its right-skewness, a nuanced outcome hidden within a stable Gini.

To understand the transmission mechanisms of U.S. shocks, we study the three transmission channels of shocks using linear state-dependence. Our findings reveal that the transmission of shocks is heterogeneous. U.S. supply shocks are a broad force, with their impact mediated by all three channels. In contrast, the powerful effects of U.S. demand shocks are far more specific, concentrated in economies with the highest degree of trade integration with the US.

Looking ahead, the expansion of detailed administrative data through projects like GRID will be crucial for testing the external validity of these channel-based findings. Future work could also benefit from employing alternative methods for identifying unanticipated shocks across all countries in the sample.

% Old findings
%In summary, we show that US supply and demand shocks increase income dispersion abroad. While demand shocks have widespread impacts, supply shocks appear more selective, with larger effects concentrated in trade-linked economies. The financial channel does not appear particularly relevant on our estimations. Domestic demand shocks tend to be weaker and more transient. Unlike US supply shocks, domestic shocks reduce inequality.
%-----------------------------------------------------------------------------
% Bibliography
\newpage
\printbibliography
%-----------------------------------------------------------------------------
% Appendix
\pagebreak
\section*{Appendix}
\renewcommand{\thetable}{A\arabic{table}}
\renewcommand{\thefigure}{B\arabic{figure}}
\setcounter{figure}{0}
\setcounter{table}{0}
\subsection*{Part A: Data} \label{appendix:a}
%-----------------------------------------------------------------------------
% Table A1: Time series description
\begin{table}[H]
\captionsetup{justification=raggedright, singlelinecheck=false}
    \centering
    \caption{Real output and unemployment series used for estimation of domestic supply and demand shocks using long-run restrictions.}
    \begin{tabular}{lcc}
    \toprule
       Country & Scope & Source  \\
    \midrule
       Canada  & 1990:Q2-2019:Q3 & OECD \\
       Denmark & 1990:Q2-2019:Q3 & OECD \\
       France  & 1990:Q2-2019:Q3 & OECD \\
       Germany & 1991:Q1-2019:Q3 & OECD \\
       Italy   & 1990:Q2-2019:Q3 & OECD \\
       Mexico  & 1990:Q2-2019:Q3 & OECD \\
       Norway  & 1990:Q2-2019:Q3 & OECD \\
       Spain   & 1990:Q2-2019:Q3 & OECD \\
       Sweden  & 1990:Q2-2019:Q3 & OECD \\
       United States & 1990:Q2-2019:Q3 & FRED \\
    \bottomrule 
    \end{tabular}
    \begin{minipage}{\textwidth}
    \vspace{0.1cm}
    \footnotesize Note: own summary, all data are quarterly. For the USA, we used GDPC1 and UNRATE series. For OECD countries, we used quarterly real GDP (expenditure approach, in USD) and the quarterly unemployment rate (seasonally adjusted, working-age population). For an alternative bivariate VAR specification that is estimated for the U.S. in subsection \ref{section:sensitivity}, we use data from FRED (output) and OECD (CPI index) for 1990:Q2-2019:Q3.
    \end{minipage}
    \label{table:a1}
\end{table}

%-----------------------------------------------------------------------------
% Oleg: Tables A2-A3 moved to the main  text

%-----------------------------------------------------------------------------

% Table A4: control variables
\begin{table}[H]
\captionsetup{justification=raggedright,
singlelinecheck=false
}
\centering
\caption{Control variables used in the estimation of local projections.}
\adjustbox{max width=\textwidth}{%
\begin{tabular}{lcc}
\toprule
  \textbf{Variable} & \textbf{Source}  & \textbf{Availability}
\tabularnewline
\midrule
  NBER identified economic recessions in the US & NBER & 1990-2019\\
  De facto component of the KOF Economic Globalization index  & \textcite{Gygli2019} & 1990-2017 \\
  Labor market regulations score (Area 5) & Fraser Institute & 1990,1995,2000-2019\\
  Share of exports to the US & Own estimation based on UNCTAD & 1990-2019, with gaps \\
  Bilateral US bank claims to GDP & Own estimation based on BIS & 1990-2019, with gaps \\
  Business confidence index & OECD & 1990-2019, with gaps
\tabularnewline
\bottomrule
\end{tabular}}
\begin{minipage}[t]{\textwidth}
\vspace{0.1cm}
\footnotesize Note: own summary.\\
\end{minipage}
\label{table:a4}
\end{table}
%-----------------------------------------------------------------------------



% Table A5: panel data coverage (revised variant)
%\begin{table}[H]
%\captionsetup{justification=raggedright, singlelinecheck=false}
%    \centering
%    \caption{Availability of GRID data (baseline sample).}
%    \begin{tabular}{l c c c c}
%    \toprule
%       Country & Scope & mean Gini & mean Gini (M) & mean Gini (F) \\
%    \midrule
%       Canada   & 1990-2019  & 0.41 (0.01) & 0.40 (0.02) & 0.38 (0.01) \\ 
%       Denmark  & 1990-2016  & 0.28 (0.01) & 0.27 (0.02) & 0.25 (0.01) \\ 
%       France   & 1991-2016  & 0.34 (0.00) & 0.34 (0.01) & 0.32 (0.00) \\ 
%       Germany  & 2001-2016  & 0.40 (0.01) & 0.36 (0.02) & 0.39 (0.01) \\ 
%       Italy    & 1990-2016  & 0.36 (0.02) & 0.34 (0.02) & 0.35 (0.02) \\ 
%       Mexico   & 2005-2019  & 0.56 (0.00) & 0.57 (0.00) & 0.54 (0.01) \\ 
%       Norway   & 1993-2017  & 0.33 (0.01) & 0.31 (0.02) & 0.32 (0.01) \\ 
%       Spain    & 2005-2018  & 0.40 (0.01) & 0.39 (0.02) & 0.39 (0.01) \\ 
%       Sweden   & 1990-2016  & 0.30 (0.01) & 0.28 (0.01) & 0.28 (0.01) \\ 
%    \midrule
%       \multicolumn{1}{l}{N} & \multicolumn{4}{l}{217} \\
%    \bottomrule
%    \end{tabular}
%    
%    \vspace{0.2cm}
%    \raggedright\footnotesize Note: standard deviations in parenthesis, all data are annual.
%    \label{table:a5}
%\end{table}
%-----------------------------------------------------------------------------

% Table A6: within country shock correlations

% \begin{table}[H]
% \captionsetup{justification=raggedright,
% singlelinecheck=false
% }
%     \centering
%     \caption{Correlation coefficients (within country) between supply and demand shocks.}
%     \begin{tabular}{lc}
%     \toprule
%     Country & Correlation \\
%     \midrule
%   CAN & -0.00 \\ 
%   DKK & 0.00 \\ 
%   DEU & 0.00 \\ 
%   ESP & -0.00 \\ 
%   FRA & -0.00 \\ 
%   ITA & 0.00 \\ 
%   MEX & -0.00 \\ 
%   NOR & 0.00 \\ 
%   SWE & -0.00 \\ 
%   USA & -0.00 \\ 
%     \bottomrule
%     \end{tabular}
    
%     \begin{minipage}{9.8cm}
%     \vspace{0.1cm}
%     \footnotesize Note: own summary, shocks are obtained using long-run restrictions. The period under analysis is 1990:Q2-2019:Q3 for all countries except Germany (1991:Q2-2019:Q3).\\  
%     \end{minipage}
%     \label{table:a6}
% \end{table}

% Table A6 - shock statistics

%\begin{table}[ht]
%\centering
%\caption{Summary statistics for demand and supply shocks.}
%\label{table:a6}
%\begin{tabular}{lrrrr}
%  \toprule
%  \multicolumn{5}{c}{\textbf{Before standardization}} \\
%  \midrule
%Statistic & US demand & US supply & Domestic demand & Domestic supply \\
%\midrule
%St. dev. & 0.49 & 0.66 & 0.56 & 0.55 \\
%Mean     & 0.00 & -0.06 & 0.02 & 0.02 \\
%Max      & 0.83 & 0.79 & 4.03 & 3.43 \\
%Min      & -0.75 & -2.36 & -2.01 & -1.73 \\
%\midrule
%\multicolumn{5}{c}{\textbf{After standardization}} \\
%\midrule
%Statistic & US demand & US supply & Domestic demand & Domestic supply \\
%\midrule
%St. dev. & 1.00 & 1.00 & 1.00 & 1.00 \\
%Mean     & 0.00 & 0.00 & 0.00 & 0.00 \\
%Max      & 1.70 & 1.29 & 7.16 & 6.19 \\
%Min      & -1.54 & -3.48 & -3.62 & -3.17 \\
%\bottomrule
%\end{tabular}
%\begin{minipage}{\textwidth}
%\vspace{0.1cm}
%\footnotesize Note: own calculations, all data are annual.
%\end{minipage}
%\end{table}

%-----------------------------------------------------------------------------
% Appendix B for empirical output
\pagebreak
\subsection*{Part B: Local projections and additional results} \label{appendix:b}
\renewcommand{\thetable}{B\arabic{table}}
\setcounter{table}{0}

% Different lags
\begin{figure}[H]
    \caption{Cumulative impulse responses to demand and supply shocks: Gini, extended horizon.}
    \label{fig:irf_longer}
    \centering
    \includegraphics[width=0.75\textwidth]{Figures/baseline_demand_supply_LP_hor_5.pdf}
    \caption*{Note: shaded areas represent 68\% Driscoll-Kraay confidence bands.}
\end{figure}

% Robustness
\begin{figure}[H]
    \centering
    \caption{Cumulative impulse responses to demand and supply shocks: Gini, all controls.}
    \label{fig:demand_supply_robust}
    \includegraphics[width=0.75\textwidth]{Figures/robust_demand_supply_LP_extended.pdf}
    \centering \caption*{Note: shaded areas represent 68\% Driscoll-Kraay confidence bands.}
\end{figure}

% Standard deviation with controls
\begin{figure}[H]
    \centering    
    \caption{Cumulative impulse responses to demand and supply shocks: standard deviation, all controls.}  
    \label{fig:std_robust}
    \includegraphics[width=0.75\textwidth]{Figures/std_demand_supply_LP_robust.pdf}
    \centering \caption*{Note: shaded areas represent 68\% Driscoll-Kraay confidence bands.}
\end{figure}

% Kelley with controls
\begin{figure}[H]
    \centering    
    \caption{Cumulative impulse responses to demand and supply shocks: Kelley skewness, all controls.}  
    \label{fig:kelley_robust}
    \includegraphics[width=0.75\textwidth]{Figures/kelley_demand_supply_LP_robust.pdf}
    \centering \caption*{Note: shaded areas represent 68\% Driscoll-Kraay confidence bands.}
\end{figure}

% 90-10 difference
\begin{figure}[H]
    \centering
    \caption{Cumulative impulse responses to demand and supply shocks: 90-10 percentile difference.}
    \label{fig:9010_demand_supply}
    \includegraphics[width=0.75\textwidth]{Figures/p9010_demand_supply_LP.pdf}
    \centering \caption*{Note: shaded areas represent 68\% Driscoll-Kraay confidence bands.}
\end{figure}
%-----------------------------------------------------------------------------
% Gender robust
%\begin{figure}[H]
%    \centering    
%    \caption{Cumulative impulse responses to demand and supply shocks: Gini (by gender), robustness.}    
%    \label{fig:demand_supply_gender_robust}
%    \includegraphics[width=0.80\textwidth]{Figures/robust_gender_LP_extended.pdf}
%    \centering \caption*{Note: grey response represents men, pink response represents women, shaded areas are 68\% Driscoll-Kraay confidence bands.}
%\end{figure}
%-----------------------------------------------------------------------------
% Table - baseline result
\newpage
\begin{table}[!htbp] 
   \centering 
    \caption{Baseline estimation results from local projections, 1990-2019.} 
    \label{tab:a7_baseline_reg_output} 
\resizebox{\textwidth}{!}{%
\begin{tabular}{@{\extracolsep{5pt}}l cccccccc} 
\\[-1.8ex]\hline 
\hline \\[-1.8ex] 
& \multicolumn{8}{c}{\textit{Dependent variable: log (Gini)}} \\ 
\hline \\[-1.8ex]

& \multicolumn{4}{c}{Demand} & \multicolumn{4}{c}{Supply} \\ 
\cmidrule(lr){2-5} \cmidrule(lr){6-9}
& (0) & (1) & (2) & (3) & (0) & (1) & (2) & (3) \\  
\hline \\[-1.8ex] 
%--------------------------------------------------------------------------- 
% US shocks - verified
%--------------------------------------------------------------------------- 
\multicolumn{9}{l}{\textit{Model 1:} US shocks} \\[1em]

Shock & $-0.001$ & $0.0002$ & $0.002$ & $0.006^{***}$ & $0.002^{**}$ & $0.004^{***}$ & $0.003^{*}$ & $0.004^{*}$ \\
 & (0.001) & (0.002) & (0.002) & (0.002) & (0.001) & (0.001) & (0.002) & (0.002) \\ [0.5em]
 
Shock$_{t-1}$  & $0.003^{***}$ & $0.005^{***}$ & $0.006^{***}$ & $0.004$ & $0.003^{***}$ & $0.003$ & $0.004$ & $0.005^{**}$ \\
 & (0.001) & (0.001) & (0.002) & (0.002) & (0.001) & (0.002) & (0.003) & (0.002) \\ [0.5em]

Shock$_{t-2}$ & $0.003^{*}$ & $0.002$ & $0.003$ & $0.003^{**}$ & $0.001$ & $0.002$ & $0.003$ & $0.001$ \\
 & (0.001) & (0.002) & (0.003) & (0.002) & (0.002) & (0.002) & (0.002) & (0.002) \\ [0.5em]

$\Delta$ Gini$_{t-1}$ & $-0.001$ & $-0.0001$ & $-0.001$ & $-0.001$ & $-0.001$ & $-0.00002$ & $-0.002$ & $-0.001$ \\ 
  & (0.001) & (0.002) & (0.002) & (0.002) & (0.001) & (0.002) & (0.002) & (0.002) \\  [0.5em]

$\Delta$ Gini$_{t-2}$  & $0.001$ & $0.0004$ & $0.001$ & $-0.001$ & $0.001$ & $0.0004$ & $0.0005$ & $-0.0002$ \\ 
  & (0.001) & (0.001) & (0.002) & (0.002) & (0.001) & (0.001) & (0.002) & (0.002) \\  [1em] 
%--------------------------------------------------------------------------- 
% Domestic shocks - verified
%--------------------------------------------------------------------------- 
\multicolumn{9}{l}{\textit{Model 2:} Domestic shocks} \\[1em]

Shock & $-0.0002$ & $0.003^{*}$ & $0.003^{*}$ & $0.0004$ & $-0.0004$ & $-0.0005$ & $-0.0005$ & $-0.0005$ \\
 & (0.001) & (0.002) & (0.002) & (0.002) & (0.001) & (0.001) & (0.002) & (0.002) \\ [0.5em]
 
Shock$_{t-1}$  & $0.004^{**}$ & $0.004^{**}$ & $0.003^{*}$ & $0.004^{**}$ & $-0.001$ & $-0.001$ & $-0.002$ & $-0.0001$ \\
 & (0.001) & (0.002) & (0.002) & (0.002) & (0.001) & (0.001) & (0.001) & (0.001) \\ [0.5em]
 
Shock$_{t-2}$ & $0.001$ & $0.001$ & $0.001$ & $0.002$ & $-0.0003$ & $-0.0003$ & $0.001$ & $0.001$ \\
& (0.001) & (0.001) & (0.001) & (0.001) & (0.001) & (0.001) & (0.001) & (0.002) \\ [0.5em]

$\Delta$ Gini$_{t-1}$ & $-0.001$ & $-0.00002$ & $-0.001$ & $-0.001$ & $-0.001$ & $0.0001$ & $-0.0003$ & $-0.0003$ \\ 
  & (0.001) & (0.002) & (0.002) & (0.002) & (0.001) & (0.002) & (0.002) & (0.002) \\  [0.5em]

$\Delta$ Gini$_{t-2}$ & $0.001$ & $0.0002$ & $0.001$ & $0.0003$ & $0.001$ & $0.0005$ & $0.001$ & $0.0005$ \\ 
  & (0.001) & (0.001) & (0.002) & (0.002) & (0.001) & (0.001) & (0.002) & (0.002) \\  [1em]
\\[-1.8ex] \hline \\[-1.8ex] 
N & 177 & 168 & 159 & 150 & 177 & 168 & 159 & 150 \\ 
\hline \\[-1.8ex]
\end{tabular}
}
\begin{minipage}{\textwidth}
    \vspace{0.1cm} 
    \footnotesize  Note: Driscoll-Kraay errors in parenthesis, column headers represent estimation horizons. Model 1 includes country fixed effects and NBER recession dummy. Model 2 includes country and year fixed effects. Significance levels: $^{*}$p$<$0.1, $^{**}$p$<$0.05, $^{***}$p$<$0.01.
\end{minipage}
\end{table}

%-----------------------------------------------------------------------------
\newpage
% Table - restricted and robustness coeff 
\begin{table}[!htbp] 
  \caption{The effect of supply and demand shocks on income inequality, 1990-2019.} 
  \label{table:a8}
  \centering
\resizebox{\textwidth}{!}{
\begin{tabular}{@{\extracolsep{5pt}}l cccc || cccc} 
 \hline \hline
 \\[-1.8ex] 
& \multicolumn{8}{c}{\textit{Dependent variable: log (Gini)}} \\\hline \\[-1.8ex]
& \multicolumn{4}{c}{Demand} & \multicolumn{4}{c}{Supply} \\ 
\cmidrule(lr){2-5} \cmidrule(lr){6-9}
\\[-1.8ex] 
& $\beta_t$ & $\beta_{t+1}$ & $\beta_{t+2}$ & $\beta_{t+3}$ & $\beta_t$ & $\beta_{t+1}$ & $\beta_{t+2}$ & $\beta_{t+3}$ \\ \hline
\\[-0.5em] 

% US shocks
\multicolumn{9}{l}{\textit{Panel 1}: US shocks} \\

(a) Baseline & $-0.001$ & $0.0002$ & $0.002$ & $0.006^{***}$  & $0.002^{**}$ & $0.004^{***}$ & $0.003^{*}$ & $0.004^{*}$ \\
& (0.001) & (0.002) & (0.002) & (0.002) & (0.001) & (0.001) & (0.002) & (0.002) \\
N & 177 & 168 & 159 & 150 & 177 & 168 & 159 & 150 \\ [1em]

(b) Restricted sample &  0.002 & 0.002 & 0.005 & 0.014$^{***}$ & 0.001 & 0.004 & $-0.0003$ & $-0.002$ \\
& (0.003) & (0.004) & (0.003) & (0.004) & (0.002) & (0.003) & (0.003) & (0.002) \\
N & 118 & 109 & 100 & 91 & 118 & 109 & 100 & 91 \\ [1em]

(c) All controls  &  $-0.0002$ & 0.002 & 0.006 & 0.016$^{**}$ & $-0.003$ & $-0.002$ & $-0.006$ & $-0.005$ \\
& (0.003) & (0.005) & (0.006) & (0.007) & (0.003) & (0.005) & (0.005) & (0.006) \\
N & 118 & 109 & 100 & 91 & 118 & 109 & 100 & 91 \\ [1em]

% Domestic shocks
\multicolumn{9}{l}{\textit{Panel 2:} Domestic shocks} \\

(a) Baseline  & $-0.0002$ & 0.003$^{*}$ & 0.003$^{*}$ & 0.003 & 0.0004 & $-0.0004$ & $-0.0005$ & $-0.0005$ \\
 & (0.001) & (0.002) & (0.002) & (0.002) & (0.001) & (0.001) & (0.002) & (0.002) \\ 
N & 177 & 168 & 159 & 150 & 177 & 168 & 159 & 150 \\ [1em]

(b) Restricted sample &  0.002 & 0.008$^{**}$ & 0.011$^{***}$ & 0.010$^{**}$ &  0.002 & $-0.001$ & $-0.002$ & $-0.001$ \\
& (0.003) & (0.003) & (0.004) & (0.005) & (0.002) & (0.003) & (0.004) & (0.005) \\
N & 118 & 109 & 100 & 91 & 118 & 109 & 100 & 91 \\ [1em]

(c) All controls & 0.002 & 0.006 & 0.008$^{**}$ & 0.007$^{**}$ & 0.0003 & $-0.002$ & $-0.004$ & $-0.002$ \\
& (0.003) & (0.004) & (0.004) & (0.003) & (0.002) & (0.003) & (0.004) & (0.004) \\
N & 118 & 109 & 100 & 91 & 118 & 109 & 100 & 91 \\ \hline 

\end{tabular}
}
\begin{minipage}{\textwidth}
    \vspace{0.1cm}
    \footnotesize  Note: Driscoll-Kraay errors in parenthesis, columns headers represent estimation horizons. Baseline regressions include additional controls for: growth of Gini (2 lags), shock (2 lags). Restricted sample is computed using baseline regressions, but only including entries, for which we have complete observations for all controls used in the estimation. For all controls, we introduce (2 lags): changes in the KOF index, changes in the labor market regulations, the share of exports to the US, bilateral US bank claims to GDP, and business confidence index. Significance levels: $^{*}$p$<$0.1, $^{**}$p$<$0.05, $^{***}$p$<$0.01.
\end{minipage}
\end{table}


%-----------------------------------------------------------------------------
% VAR impluses
\newpage
\begin{figure}[H]
    \caption{Estimated impulse response functions to demand shock.}
    \label{fig:var_impulses_demand}
    \centering
    \includegraphics[width=\textwidth, height=0.9\textheight, keepaspectratio]{Figures/all_demand.pdf}
    \caption*{Note: 20 quarters, shaded areas represent 68\% confidence bands. r\_GDP and un\_rate stand for real output growth and unemployment rate.}
\end{figure}

\newpage
\begin{figure}[H]
    \caption{Estimated impulse response functions to supply shock.}
    \label{fig:var_impulses_supply}
    \centering
    \includegraphics[width=\textwidth, height=0.9\textheight, keepaspectratio]{Figures/all_supply.pdf}
    \caption*{Note: 20 quarters, shaded areas represent 68\% confidence bands. r\_GDP and un\_rate stand for real output growth and unemployment rate.}
\end{figure}
% End
\end{document}